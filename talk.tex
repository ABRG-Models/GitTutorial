\documentclass{beamer}

\usepackage{listings}

\usepackage{color}

\definecolor{codered}{rgb}{0.69,0.09,0.12}
\definecolor{codegreen}{rgb}{0,0.6,0}
\definecolor{codegray}{rgb}{0.5,0.5,0.5}
\definecolor{codepurple}{rgb}{0.58,0,0.82}
\definecolor{backcolour}{rgb}{0.9,0.9,0.9}
\definecolor{forecolour}{rgb}{0.9,0.9,0.9}

\lstdefinestyle{bashstyle}{
    backgroundcolor=\color{backcolour},
    commentstyle=\color{codered},
    keywordstyle=\color{magenta},
    numberstyle=\tiny\color{codegray},
    stringstyle=\color{codepurple},
    basicstyle=\footnotesize,
    breakatwhitespace=false,
    breaklines=true,
    captionpos=b,
    keepspaces=true,
    numbers=left,
    numbersep=5pt,
    showspaces=false,
    showstringspaces=false,
    showtabs=false,
    tabsize=4
}

\lstset{style=bashstyle}

\title{Using Git for Comp-Neuro}

\author{Seb James}
\institute{ABRG Sheffield Internal Seminar}
\date{2018/06/28}

\begin{document}

\begin{frame}
  \titlepage % makes a title like \maketitle
\end{frame}

\begin{frame}
  \frametitle{Introduction}
  This seminar is about a command-line tool called Git.

  There's going to be a lot of \alert{text} in these slides.
\end{frame}

% Columns exampe frame
\begin{frame}
  \frametitle{Cols example}
  \begin{columns}
    \begin{column}{0.5\textwidth}
      \begin{itemize}
        \item Item 1
        \item Item 1
        \item Item 1
      \end{itemize}
    \end{column}
    \begin{column}{0.5\textwidth}
      Non-itemized content
    \end{column}
  \end{columns}
\end{frame}

% Example code listing frame
\begin{frame}[fragile] % note need for fragile when using lstlisting
  \frametitle{Example code}
  \begin{lstlisting}[language=bash]
    cd ~/blah
    ls
    zip -r file.zip
  \end{lstlisting}
\end{frame}

\begin{frame}
  \frametitle{What is Git?}
  Git is a \textbf{Revision Control} or \textbf{Version Control} tool.

  Revision control has two main features:

  \begin{enumerate}
    \pause \item Revision control allows you to keep references to different versions of a
      single file \pause without having to explicitly make copies
      \pause \item Most revision control tools allow several people to work
      on the same files %\pause with (non-magic) algorithms that (usually) correctly merge changes
  \end{enumerate}
\end{frame}

\begin{frame}
  \frametitle{File revisions}
  I bet you have folders that look like this:
  \pause \begin{itemize}
      \item myCoolProgram.cpp
      \pause \item myCoolProgram\_old.cpp
      \pause \item myCoolProgram\_1.cpp
      \pause \item myCoolProgram\_thisOneWorked.cpp
      \item myCoolProgram\_whoKnowsWhatThisOneIs.cpp
  \end{itemize}
  \pause With revision control, you only have
\end{frame}

\begin{frame}
  \frametitle{Working with other people}
\end{frame}

\begin{frame}
  \frametitle{Why did someone develop Git?}
  \begin{itemize}
    \pause \item Most revision control tools have been pretty good at
    feature 1 (file versioning)
    \pause \item ...but not great at managing multiple contributions
    \pause \item That caused Linus Torvalds to use a proprietary RCS
    called BitMover from 2002 to manage the Linux code base.
    \pause \item In 2005 Linus fell out with BitMover, and Git was
    created to replace it.
    \pause \item The name \emph{git} doesn't really mean anything.
  \end{itemize}
\end{frame}

\begin{frame}
  \frametitle{What's different about Git?}
  \begin{itemize}
  \item Git is a \emph{distributed} revision control system
  \pause \item It doesn't have the classical client-server architecture...
  \pause \item ...although typically you will work with a common
  \alert{remote} repository as your \alert{upstream} source.
  \pause \item When you \alert{clone} a repository from a source, you have
    \emph{everything} (all the file history and meta-data) in the
    files you receive to become a source for someone else to clone the
    repository.
  \pause \item That means you can work on your code, making incremental
  \alert{commits} even when you don't have internet access.
  \pause \item And every copy of the repo is a backup!
  \end{itemize}
\end{frame}

\begin{frame}
  \frametitle{Git is \emph{not} github.com}
  \begin{itemize}
  \item Github is a commercial website which makes it easy to use Git
  \item bitbucket.org is an alternative
  \pause \item Generally, public hosting is free, private hosting
  incurs a fee
  \pause \item It's pretty easy to host a git repository yourself
  \pause \item But the nice web interface has made github.com very
  popular for source code hosting
  \pause \item It's now a big business; it was acquired by Microsoft in early June
  \end{itemize}
\end{frame}

\begin{frame}
  \frametitle{Why is Git good for us?}
  \begin{itemize}
  \item We don't have hundreds of people working on the same files, but...
    \pause \item We write code, so that's very natural to hold in
    revision control (also XML)
    \pause \item We have a frequent need to ``tag'' our work (e.g. to
    match up with a paper or document)
    \pause \item If you write your papers in Latex, you can include
    your paper alongside your model code in a single, public
    repository
    \pause \item Use of Github is a very effective way to share your published models with
    your peers

  \end{itemize}
\end{frame}

\begin{frame}
  \frametitle{The Git primer}
\end{frame}

\begin{frame}
  \frametitle{}
\end{frame}

\begin{frame}
  \frametitle{}
\end{frame}

\end{document}
